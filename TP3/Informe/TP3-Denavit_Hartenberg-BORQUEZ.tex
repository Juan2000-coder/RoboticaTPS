\documentclass[a4paper,12pt]{article}
\usepackage[utf8]{inputenc}
\usepackage{graphicx}
\usepackage{fancyhdr}
\usepackage{amsmath}
\usepackage{adjustbox}
\usepackage{mathtools}
\usepackage{float}
\usepackage[spanish]{babel} 
\usepackage{lastpage}
\usepackage{amssymb} % Para símbolos matemáticos adicionales
\usepackage{cleveref}
%\usepackage[none]{hyphenat}
\usepackage{array}

\usepackage{multirow}
\usepackage{textcomp}
\usepackage[left=2.5cm, right=2.5cm, top=3cm, bottom=3cm]{geometry}

\graphicspath{{Imagenes/}}

% Encabezado y pie de página
\pagestyle{fancy}
\fancyhf{}
\setlength{\headheight}{30 pt}
\renewcommand{\headrulewidth}{0.2pt}
\fancyhead[R]{\begin{tabular}{@{}l@{}}\includegraphics[scale=0.4]{escudo.PNG}\end{tabular}}
\fancyhead[L]{\begin{tabular}{@{}c@{}} \textbf{Robótica I - Año: 2024} \\ Trabajo Práctico 3: Denavit y Hartenberg \end{tabular}}


\fancyfoot[R]{\thepage}
\fancyfoot[C]{\begin{tabular}{@{}c@{}}\textbf{BORQUEZ PEREZ Juan Manuel}\\ \textbf{Legajo 13567}\end{tabular}}
\renewcommand{\footrulewidth}{0.2pt}

\begin{document}

\begin{titlepage}
    \centering
    \vspace*{5cm}
    {\Huge\bfseries Informe de Trabajo Práctico N°3}\\
    \vspace{0.2cm}
    {\Large \textbf{Denavit y Hartenberg}}\\
    \vspace{0.5cm}
    {\Large Robótica I}\\
    \vspace{0.5 cm}
    {\Large Ingeniería en Mecatrónica}\\
    \vspace{0.2 cm}
    {\Large Facultad de Ingeniería - UNCUYO}\\
    \vspace{1.5cm}
    Alumno: Juan Manuel BORQUEZ PEREZ\\
    Legajo: 13567\\
    \vfill
    {\begin{tabular}{@{}c@{}}\includegraphics[scale=0.4]{escudo.PNG}\end{tabular}}\hspace{10pt}
    %Año 2023
\end{titlepage}

\section{Ejercicio 1.}
La convención de Denavit - Hartenberg (DH) se utiliza para establecer una matriz
de transformación homogénea que describe la posición y orientación de un sistema de
referencia respecto a otro, y está formada por el producto de 4 transformaciones elementales,
2 traslaciones y 2 rotaciones. Considere que existen algunas modificaciones de la convención
original, pero en este cursado usaremos la estándar (prestar atención a las indicaciones de los
autores al momento de presentarla).

\subsection{Inciso 1.}
\textbf{Escriba de forma simbólica cada transformación elemental, indicando si es
traslación o rotación, el parámetro principal, y con respecto a qué eje se realiza}

\begin{enumerate}
    \item Rotación alrededor del eje $Z_{i-1}$ un ángulo $\theta_i$ para llevar el eje $X_{i-1}$ hasta el eje $X_i$. Corresponde con la variable articular $q_i$.
    \[Rot\left(Z_{i-1}, \theta_i\right)\]
    \item Traslación a lo largo de $Z_{i-1}$ una distancia $d_i$ desde el origen del sistema $\{S_{i-1}\}$ hasta el eje $X_i$. Corresponde con la longitud articular.
    \[Tras\left(Z_{i-1}, d_i\right)\]
    \item Traslación a lo largo del eje $X_i$ una distancia $a_i$ desde el eje $Z_{i-1}$ al eje $Z_{i}$. Corresponde con la longitud del eslabón $i$.
    \[Tras\left(X_{i}, a_i\right)\]
    \item Rotación alrededor del eje $X_{i}$ un ángulo $\alpha_i$ desde el eje $Z_{i-1}$ al eje $Z_{i}$. Corresponde con el ángulo de torsión del eslabon $i$.
    \[Rot\left(X_{i}, \alpha_i\right)\]
\end{enumerate}

\subsection{Inciso 2.}
\textbf{Escriba el producto matricial ordenado y la forma general de la matriz homogénea
que relaciona 2 sistemas consecutivos}

\[\prescript{i-1}{}{T_i} = Rot\left(Z_{i-1}, \theta_i\right) Tras\left(Z_{i-1}, d_i\right) Tras\left(X_{i}, a_i\right) Rot\left(X_{i}, \alpha_i\right)\]

\section{Ejercicio 2}
\textbf{Aplique la convención DH a los siguientes robots. Es decir, asigne adecuadamente
los sistemas de referencia y determine los 4 parámetros de cada articulación: $\theta$, $d$, $a$, $\alpha$. Realice
un esquema adecuado donde se aprecien todos los parámetros involucrados.}

\subsection{Inciso 1}

\begin{figure}[H]
    \centering
    \begin{adjustbox}{scale = 0.85, max width=\columnwidth}
        \framebox{\includegraphics{1-Ejercicio_2_1_enunciado.JPG}}
    \end{adjustbox}
    \caption{Robot planar de 3 articulaciones rotacionales (Spong 2005).}
\end{figure}

\begin{figure}[H]
    \centering
    \begin{adjustbox}{scale = 0.85, max width=\columnwidth}
        \framebox{\includegraphics{2-Ejercicio_2_1_eslabones_articulaciones.JPG}}
    \end{adjustbox}
    \caption{Identificación de eslabones y articulaciones.}
\end{figure}

\begin{figure}[H]
    \centering
    \begin{adjustbox}{scale = 0.85, max width=\columnwidth}
        \framebox{\includegraphics{3-Ejercicio_2_1_sistemas.JPG}}
    \end{adjustbox}
    \caption{Definición de los sistemas.}
\end{figure}

\begin{figure}[H]
    \centering
    \begin{adjustbox}{scale = 0.85, max width=\columnwidth}
        \framebox{\includegraphics{4-Ejercicio_2_1_DH.JPG}}
    \end{adjustbox}
    \caption{Aplicación de la convención DH.}
\end{figure}

\subsection{Inciso 2.}

%\begin{equation*}
%    \prescript{O}{}{Rot_M} = 
%    \begin{bmatrix}
%        0.500 & -0.866\\
%        0.866 & 0.500
%    \end{bmatrix}
%\end{equation*}

%\begin{figure}[H]
%    \centering
%    \begin{adjustbox}{scale = 0.85, max width=\columnwidth}
%        \framebox{\includegraphics{1-Ejercicio_1_a.pdf}}
%    \end{adjustbox}
%    \caption{Sistema O y Sistema M superpuestos con indicación de ángulo de rotación.}
%\end{figure}

\end{document}