\documentclass[a4paper,12pt]{article}
\usepackage[utf8]{inputenc}
\usepackage{graphicx}
\usepackage{fancyhdr}
\usepackage{amsmath}
\usepackage{adjustbox}
\usepackage{mathtools}
\usepackage{float}
\usepackage[spanish, es-nodecimaldot]{babel} 
\usepackage{lastpage}
\usepackage{amssymb} % Para símbolos matemáticos adicionales
\usepackage{hyperref}
\usepackage{cleveref}
%\usepackage[none]{hyphenat}
\usepackage{array}
\usepackage{listings}
\usepackage{xcolor}

\usepackage{multirow}
\usepackage{textcomp}
\usepackage[left=2.5cm, right=2.5cm, top=3cm, bottom=3cm]{geometry}

\lstset{ 
    language=Matlab,                     % El lenguaje del código
    basicstyle=\ttfamily,                % Tipo de letra
    keywordstyle=\color{blue},           % Color para palabras clave
    commentstyle=\color{green!60!black}, % Color para comentarios
    numbers=left,                        % Numeración de las líneas
    numberstyle=\tiny\color{gray},       % Estilo para los números
    stepnumber=1,                        % Mostrar número en cada línea
    tabsize=4,                           % Tamaño de tabulación
    breaklines=true,                     % Partir líneas largas
    showspaces=false,                    % No mostrar los espacios en blanco
    showstringspaces=false,              % No mostrar espacios dentro de strings
    showtabs=false,                      % No mostrar tabs
}

\graphicspath{{Imagenes/}}

% Encabezado y pie de página
\pagestyle{fancy}
\fancyhf{}
\setlength{\headheight}{30 pt}
\renewcommand{\headrulewidth}{0.2pt}
\fancyhead[R]{\begin{tabular}{@{}l@{}}\includegraphics[scale=0.4]{escudo.PNG}\end{tabular}}
\fancyhead[L]{\begin{tabular}{@{}c@{}} \textbf{Robótica I - Año: 2024} \\ Trabajo Práctico 6: Jacobiano \end{tabular}}


\fancyfoot[R]{\thepage}
\fancyfoot[C]{\begin{tabular}{@{}c@{}}\textbf{BORQUEZ PEREZ Juan Manuel}\\ \textbf{Legajo 13567}\end{tabular}}
\renewcommand{\footrulewidth}{0.2pt}

\begin{document}

\begin{titlepage}
    \centering
    \vspace*{5cm}
    {\Huge\bfseries Informe de Trabajo Práctico N°6}\\
    \vspace{0.2cm}
    {\Large \textbf{Jacobiano}}\\
    \vspace{0.5cm}
    {\Large Robótica I}\\
    \vspace{0.5 cm}
    {\Large Ingeniería en Mecatrónica}\\
    \vspace{0.2 cm}
    {\Large Facultad de Ingeniería - UNCUYO}\\
    \vspace{1.5cm}
    Alumno: Juan Manuel BORQUEZ PEREZ\\
    Legajo: 13567\\
    \vfill
    {\begin{tabular}{@{}c@{}}\includegraphics[scale=0.4]{escudo.PNG}\end{tabular}}\hspace{10pt}
    %Año 2023
\end{titlepage}

\section{Ejercicio 6}


\[
    A = d_{5}\,C\left(2\,q_{3}+q_{4}\right)-d_{5}\,C_4-a_{3}+a_{3}\,C\left(2\,q_{3}\right)
\]
\[
    B = d_{5}\,S_4-2\,a_{2}\,S_3-a_{3}\,S\left(2\,q_{3}\right)-d_{5}\,S\left(2\,q_{3}+q_{4}\right)
\]
\[ 
    A = C\left(2\,q_{3}\right)-C_{a}\,C_4+C_a\,C\left(2\,q_{3}+q_{4}\right)-1
\]
\[
    B = S_g\,S_4-2\,S_3-S_g\,S\left(2\,q_{3}+q_{4}\right)-S_b\,S\left(2\,q_{3}\right)
\]
Vamos a descomponer paso a paso el problema que planteaste:

Expresión Original:
\[
A = \cos(2q_3) - C_a \cos(q_4) + C_a \cos(2q_3 + q_4) - 1
\]

Paso 1: Expansión de \(\cos(2q_3 + q_4)\)

Usamos la identidad de suma de ángulos:
\[
\cos(2q_3 + q_4) = \cos(2q_3)\cos(q_4) - \sin(2q_3)\sin(q_4)
\]

Entonces, la ecuación queda:
\[
A = \cos(2q_3) - C_a \cos(q_4) + C_a (\cos(2q_3)\cos(q_4) - \sin(2q_3)\sin(q_4)) - 1
\]

Paso 2: Agrupación de términos con \(\cos(q_4)\)

Reagrupamos los términos que tienen \(\cos(q_4)\):
\[
A = (\cos(2q_3) - 1) + C_a (\cos(2q_3)\cos(q_4) - \sin(2q_3)\sin(q_4)) - C_a \cos(q_4)
\]

Agrupamos \(C_a \cos(q_4)\):
\[
A = (\cos(2q_3) - 1) + C_a \cos(q_4)(\cos(2q_3) - 1) - C_a \sin(2q_3)\sin(q_4)
\]

Paso 3: Reemplazo de \(1 - \cos(2q_3)\) por \(2\sin^2(q_3)\)

Aplicamos la identidad trigonométrica:
\[
1 - \cos(2q_3) = 2\sin^2(q_3)
\]

Reemplazamos en la ecuación:
\[
A = -2\sin^2(q_3) + C_a \cos(q_4)(-2\sin^2(q_3)) - C_a \sin(2q_3)\sin(q_4)
\]

Resultado Final:
\[
A = -2\sin^2(q_3) - 2C_a \cos(q_4)\sin^2(q_3) - C_a \sin(2q_3)\sin(q_4)
\]

Este sería el resultado final con los términos correctamente agrupados y el reemplazo realizado.

Vamos a seguir los pasos que indicaste, comenzando desde la expresión final que obtuvimos anteriormente.

Expresión Original:

\[
A = -2\sin^2(q_3) - 2C_a \cos(q_4)\sin^2(q_3) - C_a \sin(2q_3)\sin(q_4)
\]

Paso 1: Expansión de \(\sin(2q_3)\)

Usamos la identidad de doble ángulo para el seno:
\[
\sin(2q_3) = 2\sin(q_3)\cos(q_3)
\]

Sustituyendo en la ecuación:
\[
A = -2\sin^2(q_3) - 2C_a \cos(q_4)\sin^2(q_3) - C_a (2\sin(q_3)\cos(q_3))\sin(q_4)
\]

Paso 2: Dividir toda la expresión por \(\sin(q_3)\)

Dividimos cada término entre \(\sin(q_3)\):
\[
\frac{A}{\sin(q_3)} = -2\sin(q_3) - 2C_a \cos(q_4)\sin(q_3) - 2C_a \cos(q_3)\sin(q_4)
\]

Paso 3: Factorizar \(\sin(q_3)\) y \(\cos(q_3)\)

Reagrupamos para factorizar:

- Agrupamos términos con \(\sin(q_3)\):
\[
\frac{A}{\sin(q_3)} = \sin(q_3)(-2 - 2C_a \cos(q_4)) - 2C_a \cos(q_3)\sin(q_4)
\]

- Ahora factorizamos \(\sin(q_3)\) y \(\cos(q_3)\) por separado:

\[
\frac{A}{\sin(q_3)} = \sin(q_3)(-2 - 2C_a \cos(q_4)) + \cos(q_3)(-2C_a \sin(q_4))
\]

Resultado Final:

La expresión final es:
\[
A = S_3(1 + C_a C_4) + C_3(C_a S_4)
\]

Este sería el resultado con los términos factorizados en función de \(\sin(q_3)\) y \(\cos(q_3)\), dividiendo la expresión por \(\sin(q_3)\) como solicitaste.
Eso tiene que ser cero, y se puede expresar como;

\begin{equation}
    u =
    \begin{bmatrix}
        C_3\\
        S_3
    \end{bmatrix}
\end{equation}

\begin{equation}
    v =
    \begin{bmatrix}
        1 + 2C_aC_4\\
        C_aS_4
    \end{bmatrix}
\end{equation}

La expresión se reduce a:
\[u \cdot v = 0\]
Pero $\mathbf{v}$ depende solamente de $q_4$. Para cada valor del mismo
se pueden encontrar el vector $\mathbf{u}$ que indica la dirección normal.

Para B por otro lado.
Vamos a desarrollar los pasos que mencionaste para la expresión \( B \).

 Expresión original:

\[
B = S_g \sin(q_4) - 2\sin(q_3) - S_g \sin(2q_3 + q_4) - S_b \sin(2q_3)
\]

 Paso 1: Expandir \(\sin(2q_3 + q_4)\)

Usamos la identidad de suma de ángulos para el seno:

\[
\sin(2q_3 + q_4) = \sin(2q_3)\cos(q_4) + \cos(2q_3)\sin(q_4)
\]

Sustituyendo en la expresión:

\[
B = S_g \sin(q_4) - 2\sin(q_3) - S_g (\sin(2q_3)\cos(q_4) + \cos(2q_3)\sin(q_4)) - S_b \sin(2q_3)
\]

 Paso 2: Agrupar los términos con \( S_g \sin(q_4) \)

Reagrupamos los términos que tienen \( \sin(q_4) \):

\[
B = S_g \sin(q_4) - S_g \cos(2q_3)\sin(q_4) - 2\sin(q_3) - S_g \sin(2q_3)\cos(q_4) - S_b \sin(2q_3)
\]

Factorizamos \( S_g \sin(q_4) \):

\[
B = S_g \sin(q_4)(1 - \cos(2q_3)) - 2\sin(q_3) - S_g \sin(2q_3)\cos(q_4) - S_b \sin(2q_3)
\]

 Paso 3: Agrupar los términos con \( \sin(2q_3) \)

Reagrupamos los términos que contienen \( \sin(2q_3) \):

\[
B = S_g \sin(q_4)(1 - \cos(2q_3)) - 2\sin(q_3) - (\sin(2q_3))(S_g \cos(q_4) + S_b)
\]

 Paso 4: Reemplazar \( 1 - \cos(2q_3) \) por \( 2\sin^2(q_3) \)

Usamos la identidad trigonométrica \( 1 - \cos(2q_3) = 2\sin^2(q_3) \) en el primer término:

\[
B = S_g \sin(q_4)(2\sin^2(q_3)) - 2\sin(q_3) - (\sin(2q_3))(S_g \cos(q_4) + S_b)
\]

 Paso 5: Expandir \( \sin(2q_3) \)

Usamos la identidad \( \sin(2q_3) = 2\sin(q_3)\cos(q_3) \):

\[
B = S_g \sin(q_4)(2\sin^2(q_3)) - 2\sin(q_3) - 2\sin(q_3)\cos(q_3)(S_g \cos(q_4) + S_b)
\]

 Paso 6: Dividir toda la expresión por \( \sin(q_3) \)

Dividimos cada término entre \( \sin(q_3) \):

\[
\frac{B}{\sin(q_3)} = S_g \sin(q_4)(2\sin(q_3)) - 2 - 2\cos(q_3)(S_g \cos(q_4) + S_b)
\]

 Resultado final:

La expresión final es:

\[
B= S_g S_3S_4 - 1 - C_3(S_g C_4 + S_b)
\]

Ahora: 
\begin{equation}
    w =
    \begin{bmatrix}
        -S_gC_4 - S_b\\
        S_gS_4
    \end{bmatrix}
\end{equation}

La expresión se reduce a:
\[u \cdot w = 1\]

Esto es $\parallel u \parallel \parallel w \parallel \cos(\zeta) = 1$
Pero $\parallel u \parallel  = 1$. Luego
\[
    \cos(\zeta) = \frac{1}{\parallel w \parallel}
\]
Eso implica que debe ser:
\[
    \parallel(w) \parallel > 1
\]

\[
\|w\|^2 = S_g^2 + 2S_g S_b C_4 + S_b^2 > 1
\]

Esta es la expresión del cuadrado del módulo del vector \( w \).
Para los valores de q4 que se cumple a desigualdad anterior se obtienen
el vector $v$ y el vector $w$ que necesariamente deben estar alineados para que
determinen un único valor de $u$ y por lo tanto de $q_3$.

Eso último todavía hay que verificarlo.


Otra forma de hacerlo mas rápido es:
Dado que los vectores \( w \) y \( v \) son:

\[
w = \begin{bmatrix} -S_g C_4 - S_b \\ S_g S_4 \end{bmatrix}
\]
\[
v = \begin{bmatrix} 1 + 2C_a C_4 \\ C_a S_4 \end{bmatrix}
\]

Queremos formar la matriz del sistema de ecuaciones \( T \) usando estos vectores como filas:

\[
\begin{bmatrix} C_a S_4 & 1 + C_a C_4 \\ -S_g C_4 - S_b & S_g S_4 \end{bmatrix}
\cdot
\begin{bmatrix}C_3\\S_3\end{bmatrix}
=
\begin{bmatrix}0\\1\end{bmatrix}
\]

\end{document}