\documentclass[a4paper,12pt]{article}
\usepackage[utf8]{inputenc}
\usepackage{graphicx}
\usepackage{fancyhdr}
\usepackage{amsmath}
\usepackage{adjustbox}
\usepackage{mathtools}
\usepackage{float}
\usepackage[spanish, es-nodecimaldot]{babel} 
\usepackage{lastpage}
\usepackage{amssymb} % Para símbolos matemáticos adicionales
\usepackage{hyperref}
\usepackage{cleveref}
%\usepackage[none]{hyphenat}
\usepackage{array}
\usepackage{listings}
\usepackage{xcolor}

\usepackage{multirow}
\usepackage{textcomp}
\usepackage[left=2.5cm, right=2.5cm, top=3cm, bottom=3cm]{geometry}

\lstset{ 
    language=Matlab,                     % El lenguaje del código
    basicstyle=\ttfamily,                % Tipo de letra
    keywordstyle=\color{blue},           % Color para palabras clave
    commentstyle=\color{green!60!black}, % Color para comentarios
    numbers=left,                        % Numeración de las líneas
    numberstyle=\tiny\color{gray},       % Estilo para los números
    stepnumber=1,                        % Mostrar número en cada línea
    tabsize=4,                           % Tamaño de tabulación
    breaklines=true,                     % Partir líneas largas
    showspaces=false,                    % No mostrar los espacios en blanco
    showstringspaces=false,              % No mostrar espacios dentro de strings
    showtabs=false,                      % No mostrar tabs
}

\graphicspath{{Imagenes/}}

% Encabezado y pie de página
\pagestyle{fancy}
\fancyhf{}
\setlength{\headheight}{30 pt}
\renewcommand{\headrulewidth}{0.2pt}
\fancyhead[R]{\begin{tabular}{@{}l@{}}\includegraphics[scale=0.4]{escudo.PNG}\end{tabular}}
\fancyhead[L]{\begin{tabular}{@{}c@{}} \textbf{Robótica I - Año: 2024} \\ Trabajo Práctico 6: Jacobiano \end{tabular}}


\fancyfoot[R]{\thepage}
\fancyfoot[C]{\begin{tabular}{@{}c@{}}\textbf{BORQUEZ PEREZ Juan Manuel}\\ \textbf{Legajo 13567}\end{tabular}}
\renewcommand{\footrulewidth}{0.2pt}

\begin{document}

\begin{titlepage}
    \centering
    \vspace*{5cm}
    {\Huge\bfseries Informe de Trabajo Práctico N°6}\\
    \vspace{0.2cm}
    {\Large \textbf{Jacobiano}}\\
    \vspace{0.5cm}
    {\Large Robótica I}\\
    \vspace{0.5 cm}
    {\Large Ingeniería en Mecatrónica}\\
    \vspace{0.2 cm}
    {\Large Facultad de Ingeniería - UNCUYO}\\
    \vspace{1.5cm}
    Alumno: Juan Manuel BORQUEZ PEREZ\\
    Legajo: 13567\\
    \vfill
    {\begin{tabular}{@{}c@{}}\includegraphics[scale=0.4]{escudo.PNG}\end{tabular}}\hspace{10pt}
    %Año 2023
\end{titlepage}

\section{Ejercicio 1}
\begin{figure}[H]
    \centering
    \begin{adjustbox}{scale = 0.55, max width=\columnwidth}
        \framebox{\includegraphics{1-Ejercicio_1.JPG}}
    \end{adjustbox}
    \caption{Robot planar RR Ejercicio 1.}
\end{figure}

\subsection{Mediante derivación respecto del tiempo obtenga el Jacobiano del robot}
Se tiene:

\begin{equation}
    \left\{
    \begin{aligned}
    x &= a_1 \cos(\theta_1) + a_2 \cos(\theta_1 + \theta_2) \\
    y &= a_1 \sin(\theta_1) + a_2 \sin(\theta_1 + \theta_2) \\
    z &= 0 \\
    \alpha &= 0 \\
    \beta &= 0 \\
    \gamma &= \theta_1 + \theta_2
    \end{aligned}
    \right.
    \label{directa RR}
\end{equation}

Cada fila en el Jacobiano se puede obtener como el gradiente de la función en esa fila
respecto de las variables $q_1\equiv\theta_1$ y $q_2\equiv\theta_2$. Se obtiene:

\begin{equation}
    J(q) = 
    \begin{bmatrix}
        -a_1\sin(\theta_1) - a_2\sin(\theta_1 + \theta_2) & -a_2\sin(\theta_1 + \theta_2)\\
        a_1\cos(\theta_1) + a_2\cos(\theta_1 + \theta_2)  & a_2\cos(\theta_1 + \theta_2)\\
        0                                                 & 0\\
        0                                                 & 0\\
        0                                                 & 0\\
        1                                                 & 1
    \end{bmatrix}
    \label{jacobiano RR}
\end{equation}

\subsection{Calcule la velocidad del extremo $\dot{p}$ , en m/s para $q = [\pi/6, \pi/6]$, en rad y $\dot{q}=
[0, -1]$ en rad/s. Suponga longitud de eslabón unitaria. Observe el gráfico del robot
e interprete los resultados}

Las velocidades en el extremo se calculan a partir de las velocidades articulares como:
\begin{equation}
    \dot{p} = J(q)\dot{q}
\end{equation}

Para los valores dados se obtiene:
\begin{equation*}
    \dot{p} = 
    \begin{bmatrix}
        -\sin(\pi/6) - \sin(\pi/3) & -\sin(\pi/3)\\
        \cos(\pi/6) + \cos(\pi/3)  & \cos(\pi/3)\\
        0                          & 0\\
        0                          & 0\\
        0                          & 0\\
        1                          & 1
    \end{bmatrix}
    \begin{bmatrix}
        0\\
        -1
    \end{bmatrix}
    =
    \begin{bmatrix}
        \sqrt{3}/2\\
        -1/2\\
        0\\
        0\\
        0\\
        -1
    \end{bmatrix}
    =
    \begin{bmatrix}
        0.866\\
        -0.500\\
        0.000\\
        0.000\\
        0.000\\
        -1.000
    \end{bmatrix}
\end{equation*}

En donde las primeras 3 componentes están en $m/s$ y las últimas 3 en $rad/s$.

\subsection{Trabaje solo con las coordenadas X-Y (primeras 2 filas del J) y verifique mediante la
inversa algebraica que $\dot{q} = J^{-1}(q)\dot{p}$ se cumple}

Conservando solo las filas del plano X-Y nos queda:

\begin{equation}
    J(q) = 
    \begin{bmatrix}
        -a_1\sin(\theta_1) - a_2\sin(\theta_1 + \theta_2) & -a_2\sin(\theta_1 + \theta_2)\\
        a_1\cos(\theta_1) + a_2\cos(\theta_1 + \theta_2)  & a_2\cos(\theta_1 + \theta_2)
    \end{bmatrix}
\end{equation}

Luego $J^{-1}(q)$ es:
\begin{equation}
    J^{-1}(q) =
    \begin{bmatrix}
        \frac{\cos(\theta_1 + \theta_2)}{a_1 \sin(\theta_2)} & \frac{\sin(\theta_1 + \theta_2)}{a_1 \sin(\theta_2)} \\
        -\frac{a_1 \cos(\theta_1) + a_2 \cos(\theta_1 + \theta_2)}{a_1 a_2 \sin(\theta_2)} & -\frac{a_1 \sin(\theta_1) + a_2 \sin(\theta_1 + \theta_2)}{a_1 a_2 \sin(\theta_2)}
    \end{bmatrix}
\end{equation}

Para los valores del inciso anterior verificamos:
\begin{equation*}
    J^{-1}(q)\dot{p} =
    \begin{bmatrix}
        \frac{\cos(\pi/3)}{\sin(\pi/6)} & \frac{\sin(\pi/3)}{\sin(\pi/6)} \\
        -\frac{\cos(\pi/6) + \cos(\pi/3)}{\sin(\pi/6)} & -\frac{\sin(\pi/6) + \sin(\pi/3)}{\sin(\pi/6)}
    \end{bmatrix}
    \begin{bmatrix}
        \sqrt{3}/2\\
        -1/2
    \end{bmatrix}
    = 
    \begin{bmatrix}
        0\\
        -1
    \end{bmatrix}
    = \dot{q}
\end{equation*}

\section{Ejercicio 3}
\textbf{halle el Jacobiano en forma general de los 3 robots siguientes}
\subsection{}

\begin{figure}[H]
    \centering
    \begin{adjustbox}{scale = 0.55, max width=\columnwidth}
        \framebox{\includegraphics{2-Ejercicio_2_1_RR_DH.JPG}}
    \end{adjustbox}
    \caption{Convención DH utilizada para robot RRR planar.}
\end{figure}

Las ecuaciones de la cinemática directa se obtienen extendiendo las indicadas en \cref{directa RR} y se obtiene
\begin{equation*}
    \left\{
    \begin{aligned}
    x &= a_1 \cos(\theta_1) + a_2 \cos(\theta_1 + \theta_2) + a_3 \cos(\theta_1 + \theta_2 + \theta_3)\\
    y &= a_1 \sin(\theta_1) + a_2 \sin(\theta_1 + \theta_2) + a_3 \sin(\theta_1 + \theta_2 + \theta_3)\\
    z &= 0 \\
    \alpha &= 0 \\
    \beta &= 0 \\
    \gamma &= \theta_1 + \theta_2 + \theta_3
    \end{aligned}
    \right.
    \label{directa RRR}
\end{equation*}

El jacobiano se obtiene también por extensión de \cref{jacobiano RR}
\begin{equation}
    J(q) = 
    \begin{bmatrix}
        -a_1S_1 - a_2S_{12} - a_3S_{123}& -a_2S_{12} - a_3S_{123} & - a_3S_{123}\\
        a_1C_1 + a_2C_{12} + a_3C_{123}& a_2C_{12} + a_3C_{123}  &  a_3C_{123}\\
        0                               & 0                       & 0\\
        0                               & 0                       & 0\\
        0                               & 0                       & 0\\
        1                               & 1                       & 1
    \end{bmatrix}
    \label{jacobiano RRR}
\end{equation}

\subsection{}
\begin{figure}[H]
    \centering
    \begin{adjustbox}{scale = 0.55, max width=\columnwidth}
        \framebox{\includegraphics{3-Ejercicio_2_1_RLR_DH.JPG}}
    \end{adjustbox}
    \caption{Convención DH utilizada para robot RLR planar.}
\end{figure}

Los parámetros de DH del robot son:
\begin{table}[H]
    \centering
    \begin{tabular}{|c|c|c|c|c|c|}
    \hline
    Sistema & $\theta$          & $d$         & $a$         & $\alpha$     & $\sigma$ \\ \hline
    1       & $q_1$             & 0           & $0$         & $\pi/2$   & 0        \\ \hline
    2       & $0$               & $q_2$       & $0$         & $-\pi/2$  & 1        \\ \hline
    3       & $q_3$             & 0           & $a_3$  & $0$          & 0        \\ \hline
    \end{tabular}
    \caption{Parámetros DH RLR planar.}
\end{table}

Las ecuaciones de la cinemática directa se obtienen aplicando ``fkine'' de forma simbólica
en MATLAB:

\begin{equation*}
    CD = 
    \left[\begin{array}{cccc}
        \cos\left(q_{1}+q_{3}\right) & -\sin\left(q_{1}+q_{3}\right) & 0 & a_{3}\,\cos\left(q_{1}+q_{3}\right)+q_{2}\,\sin\left(q_{1}\right)\\
        \sin\left(q_{1}+q_{3}\right) & \cos\left(q_{1}+q_{3}\right) & 0 & a_{3}\,\sin\left(q_{1}+q_{3}\right)-q_{2}\,\cos\left(q_{1}\right)\\
        0 & 0 & 1 & 0\\
        0 & 0 & 0 & 1
    \end{array}
    \right]
\end{equation*}

Luego tenemos:
\begin{equation*}
    \left\{
    \begin{aligned}
    x &= a_{3}\,\cos\left(q_{1}+q_{3}\right)+q_{2}\,\sin\left(q_{1}\right)\\
    y &= a_{3}\,\sin\left(q_{1}+q_{3}\right)-q_{2}\,\cos\left(q_{1}\right)\\
    z &= 0 \\
    \alpha &= 0 \\
    \beta &= 0 \\
    \gamma &= q_1 + q_3
    \end{aligned}
    \right.
\end{equation*}

Finalmente el jacobiano es:

\begin{equation}
    J(q) = 
    \begin{bmatrix}
        -a_3\sin(q_1 + q_3) + q_2\cos(q_1) & \sin(q_1)  & -a_3\sin(q_1 + q_3)\\
        a_3\cos(q_1 + q_3)  + q_2\sin(q_1) & -\cos(q_1) & a_3\cos(q_1 + q_3)\\
        0 & 0 & 0\\
        0 & 0 & 0\\
        0 & 0 & 0\\
        1 & 0 & 1
    \end{bmatrix}
\end{equation}

\subsection{}

\begin{figure}[H]
    \centering
    \begin{adjustbox}{scale = 0.55, max width=\columnwidth}
        \framebox{\includegraphics{4-Ejercicio_2_2_LRR_DH.JPG}}
    \end{adjustbox}
    \caption{Convención DH utilizada para robot LRR}
\end{figure}

\begin{table}[H]
    \centering
    \begin{tabular}{|c|c|c|c|c|c|}
    \hline
    Sistema & $\theta$  & $d$ & $a$         & $\alpha$ & $\sigma$ \\ \hline
    1       & $0$       & $q_1$ & $a_{1}$  & 0      & 1        \\ \hline
    2       & $q_2$     & 0   & $a_{2}$  & 0        & 0        \\ \hline
    3       & $q_3$     & 0   & $a_{3}$  & 0        & 0        \\ \hline
    \end{tabular}
    \caption{Parámetros DH robot LRR}
\end{table}

La cinemática directa obtenida de forma simbólica con MATLAB es:
\begin{equation*}
    CD = \left[\begin{array}{cccc} \cos\left(q_{2}+q_{3}\right) & -\sin\left(q_{2}+q_{3}\right) & 0 & a_{1}+a_{3}\,\cos\left(q_{2}+q_{3}\right)+a_{2}\,\cos\left(q_{2}\right)\\ \sin\left(q_{2}+q_{3}\right) & \cos\left(q_{2}+q_{3}\right) & 0 & a_{3}\,\sin\left(q_{2}+q_{3}\right)+a_{2}\,\sin\left(q_{2}\right)\\ 0 & 0 & 1 & q_{1}\\ 0 & 0 & 0 & 1 \end{array}\right]
\end{equation*}

De donde se obtiene
\begin{equation*}
    \left\{
        \begin{aligned}
            x &= a_1 +  a_2\,\cos(q_2) + a_3\,\cos(q_2 + q_3)\\
            y &= a_2\,\sin(q_2) + a_3\,\sin(q_2 + q_3)\\
            z &= q_1\\
            \alpha &= 0\\
            \beta  &= 0\\
            \gamma &= q_2 + q_3
        \end{aligned}
    \right.
\end{equation*}

Luego el Jacobiano es:
\begin{equation}
    J(q) = 
    \begin{bmatrix}
        0 & -a_2\,\sin(q_2) - a_3\,\sin(q_2 + q_3) & -a_3\,\sin(q_2 + q_3)\\
        0 & a_2\,\cos(q_2) + a_3\,\cos(q_2 + q_3)  & a_3\,\cos(q_2 + q_3)\\
        1 & 0 & 0 \\
        0 & 0 & 0 \\
        0 & 0 & 0 \\
        0 & 1 & 1
    \end{bmatrix}
\end{equation}

\section{Ejercicio 3}
%\begin{equation*}
%    \prescript{O}{}{Rot_M} = 
%    \begin{bmatrix}
%        0.500 & -0.866\\
%        0.866 & 0.500
%    \end{bmatrix}
%\end{equation*}

%\begin{figure}[H]
%    \centering
%    \begin{adjustbox}{scale = 0.85, max width=\columnwidth}
%        \framebox{\includegraphics{1-Ejercicio_1_a.pdf}}
%    \end{adjustbox}
%    \caption{Sistema O y Sistema M superpuestos con indicación de ángulo de rotación.}
%\end{figure}


%\begin{table}[H]
%    \centering
%    \begin{tabular}{|c|c|c|c|c|c|}
%    \hline
%    Sistema & $\theta$  & $d$           & $a$    & $\alpha$ & $\sigma$ \\ \hline
%    1       & $q_1$     & $199.2$       & $200$  & 0        & 0        \\ \hline
%    2       & $q_2$     & $59.5$        & $250$  & 0        & 0        \\ \hline
%    3       & $0$       & $q_3$         & $0$    & 180°     & 1        \\ \hline
%    4       & $q_4$     & $37.5$        & $0$    & 0        & 0        \\ \hline
%    \end{tabular}
%    \caption{Parámetros DH alternativos.}
%    \label{parametros DH2}
%\end{table}

\end{document}